\begin{abstract}
A fundamental task in various science disciplines is to find underlying causal relations and use them. Finding out why an event occurs, its cause, means that we can, for example, stop the effect from happening if we remove this knowledge from the equation, or we can generate the subsequent effect if we replicate it. A traditional way to discover causal relations is to use interventions or randomized experiments, which is, however, in many cases too expensive, too time-consuming, unethical, or even impossible. As a result, causal knowledge can be derived from observational data in a purely data-driven manner. This paper addresses the challenges in estimating the causal generating processes for time-series data, which helps researchers in different fields, from its use in medical research to climate and cloud computing, among many others. This paper aims to provide an introduction and a brief review of the computational methods for causal discovery, including constraint-based, score-based methods and those based on functional causal models.
\keywords{Conditional Independence \and Causal Discovery \and  Network logs}
\end{abstract}